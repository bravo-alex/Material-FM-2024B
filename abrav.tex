% -- Formato para ejemplos
\newtheoremstyle{estiloejemplo}%
    {9pt}{9pt}%
    {}%
    {0pt}%
    {\bfseries\scshape}{.}%
    { }%
    {}

% -- Ambiente para ejemplos
\theoremstyle{estiloejemplo}
    \newtheorem{ejemplo}{Ejemplo}
    \newtheorem*{see}{Observación}
\newtheorem{exer}{Ejercicio}[]

% -- Formato para teoremas
\newtheoremstyle{estiloteorema}%
    {9pt}{9pt}%
    {\slshape }%\slshape
    {0pt}%
    {\bfseries \scshape }{.}%
    { }%\scshape
    {}

% -- Ambiente para teoremas
\theoremstyle{estiloteorema}
    \newtheorem{defi}{Definición}[section]
    \newtheorem{axioma}{Axioma}
    \newtheorem{teorema}{Teorema}[section]
    \newtheorem{corolario}{Corolario}[section]
    \newtheorem{proposicion}[teorema]{Proposición}
    \newtheorem{lema}{Lema}
    \newtheorem{notacion}{Notaci\'on}
%\newtheorem{lema}[teorema]{Lema}
\newcommand{\encabezado}{\begin{figure}
    \includegraphics[scale=0.012]{epn.png} \hfill
   \includegraphics[scale=0.045]{facu.png}
\end{figure}
\vspace*{-1.8cm}
\hspace{1cm}{{\textbf{Análisis Real }}} \hspace{6cm}  \textit{\textbf{Semestre:} 2021-A}
\vspace*{0.43cm}
{\hrule height 1.4pt}
\begin{center}
    {\textbf{Taller No.1}}
\end{center} 
\textbf{Nombre:} Alex Wladimir Bravo Quisaguano \hfill \textbf{Fecha:} \today}
\allowdisplaybreaks 